\usepackage{standalone}
\usepackage{tikz}
\usetikzlibrary{arrows,automata,backgrounds,calc,external,matrix,positioning}
\usepackage[linguistics]{forest}

\newcommand{\inputmid}[1]{\input{#1}}
\newcommand{\inputlarge}[1]{\input{#1}}

\usepackage[charter]{mathdesign}
\usepackage[amsmath,hyperref,thmmarks]{ntheorem}

\theoremstyle{plain}
\theoremheaderfont{\normalfont\bfseries}
\theorembodyfont{\normalfont\upshape}
\theoremseparator{.}
\newcommand{\TheoremSymbol}{\tikzset{external/export next=false}\tikz{\draw (0,0) -| (.5em,.5em);}}
\theoremsymbol{\ensuremath{\TheoremSymbol}}
\newtheorem{theorem}{Theorem}
\newtheorem{lemma}[theorem]{Lemma}
\newtheorem{proposition}[theorem]{Proposition}
\newtheorem{corollary}[theorem]{Corollary}
\theorembodyfont{\normalfont\itshape}
\theoremsymbol{}
\newtheorem{conjecture}[theorem]{Conjecture}
\newtheorem{problem}[theorem]{Problem}
\theorembodyfont{\normalfont\upshape}
\theoremprework{\bigskip\hrule}
\theorempostwork{\hrule\bigskip}
\newtheorem{definition}[theorem]{Definition}
\theoremstyle{nonumberplain}
\theoremheaderfont{\normalfont\itshape}
\theoremprework{}
\theorempostwork{}
\newcommand{\QED}{\ensuremath{\blacksquare}}
\qedsymbol{\QED}
\theoremsymbol{\ensuremath{\Box}}
\newtheorem{proof}{Proof}
\theoremsymbol{\ensuremath{\bigcirc}}
\newtheorem{remark}{Remark}
\newtheorem{example}{Example}
\theoremsymbol{}
\theoremheaderfont{\normalfont\scshape}
\theoremindent1.5em
\newtheorem{ucase}{Case}
\theoremstyle{plain}
\newtheorem{case}{Case}
\theoremindent1.5em
\newtheorem{subcase}{Case}
\numberwithin{subcase}{case}

